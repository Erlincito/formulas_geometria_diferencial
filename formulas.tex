\documentclass{article}

%Para reconocer tildes y otras cosas de español
\usepackage[utf8]{inputenc}

%Símbolos de uso común:
\newcommand{\RR}{\mathbb{R}}
\renewcommand{\Re}{\text{Re\,}}
\newcommand{\QQ}{\mathbb{Q}}
\newcommand{\NN}{\mathbb{N}}
\newcommand{\CC}{\mathbb{C}}
\newcommand{\ZZ}{\mathbb{Z}}
\renewcommand{\SS}{\EuScript{S}}

%Norma de un vector
\renewcommand{\v}{\Vert}

\begin{document}
\section{Introducción}
Se imitará la notación del libro \cite{edg}, de antemano se asume que todas las curvas y superficies son suaves a menos que se diga lo contrario.

\section{Curvas}
\subsection{Curvatura}
La curvatura de una curva $\gamma$ parametrizada con respecto a su longitud de arco $s$ es la segunda derivada con respecto a $s$:
$$\kappa = \v \ddot \gamma(s)\v$$

\begin{thebibliography}{9}
\bibitem{edg}
Andrew Pressley,
\emph{Elementary Differentiable Geometry},
Springer Undergraduate Mathematics Series (SUMS),
Second Edition, 
2010.
\end{thebibliography}
\end{document}
